\documentclass{article}
\usepackage[utf8]{inputenc}

\title{Tarea 4}
\author{Tonny Alberto Gualdron Pacheco}
\date{November 2018}

\usepackage{natbib}
\usepackage{graphicx}

\begin{document}

\maketitle

\section{ODE}

Movimiento de un proyectil con velocidad inicial de 300 m/s, masa 0.2 kg y una constante de fricción del aire con factor de 0.2.Se simulo el movimiento con el método runge kutta 4.\\

\begin{figure}[h!]
\centering
\includegraphics[scale=0.5]{45grados.jpg}
\caption{45 grados}
\label{fig:45}
\end{figure}

La distancia recorrida fue 4.37597 m.\\

simulación para ángulo inicial entre 10 y 70 grados.\\

\begin{figure}[h!]
\centering
\includegraphics[scale=0.5]{10-70grados.jpg}
\caption{10-70 grados}
\label{fig:10-70}
\end{figure}

Para ángulos entre 10 y 70 grados la mayor distancia recorrida es 5.37768 m que corresponde a un ángulo de 20 grados.\\

\section{PDE}

El ejercicio consiste en una barra 373.15 K insertada en una piedra calcita, la barra tiene temperatura constante de 373.15 K y se simula la difusión térmica.\\ 

\subsection{Fronteras igual a 283.15 K}

\begin{figure}[h!]
\centering
\includegraphics[scale=0.5]{Caso1t0.jpg}
\caption{Caso 1 t=0}
\label{fig:1}
\end{figure}

\begin{figure}[h!]
\centering
\includegraphics[scale=0.5]{Caso1t1.jpg}
\caption{Caso 1 t=1/4 T}
\label{fig:2}
\end{figure}

\begin{figure}[h!]
\centering
\includegraphics[scale=0.5]{Caso1t2.jpg}
\caption{Caso 1 t=3/4 T}
\label{fig:3}
\end{figure}

\begin{figure}[h!]
\centering
\includegraphics[scale=0.5]{Caso2t.jpg}
\caption{Caso 1 t= T}
\label{fig:4}
\end{figure}

Las anteriores gráficas presenta el comportamiento para condiciones de frontera constantes, es el caso con el cual se llega mas rápido al equilibrio térmico.\\

\begin{figure}[h!]
\centering
\includegraphics[scale=0.5]{Caso1Tprom.jpg}
\caption{Caso 1 temperatura promedio VS tiempo}
\label{fig:13}
\end{figure}

\subsection{Fronteras abiertas}

\begin{figure}[h!]
\centering
\includegraphics[scale=0.5]{Caso2t0.jpg}
\caption{Caso 2 t=0}
\label{fig:5}
\end{figure}

\begin{figure}[h!]
\centering
\includegraphics[scale=0.5]{Caso2t1.jpg}
\caption{Caso 2 t=1/4 T}
\label{fig:6}
\end{figure}

\begin{figure}[h!]
\centering
\includegraphics[scale=0.5]{Caso2t2.jpg}
\caption{Caso 2 t=3/4 T}
\label{fig:7}
\end{figure}

\begin{figure}[h!]
\centering
\includegraphics[scale=0.5]{Caso2t.jpg}
\caption{Caso 2 t= T}
\label{fig:8}
\end{figure}

Las anteriores gráficas presenta el comportamiento para condiciones de frontera abiertas, tiene un comportamiento similar al de fronteras constantes.\\

\begin{figure}[h!]
\centering
\includegraphics[scale=0.5]{Caso2Tprom.jpg}
\caption{Caso 2 temperatura promedio VS tiempo}
\label{fig:14}
\end{figure}

\subsection{Fronteras periódicas}

\begin{figure}[h!]
\centering
\includegraphics[scale=0.5]{Caso3t0.jpg}
\caption{Caso 3 t=0}
\label{fig:9}
\end{figure}

\begin{figure}[h!]
\centering
\includegraphics[scale=0.5]{Caso3t1.jpg}
\caption{Caso 3 t=1/4 T}
\label{fig:10}
\end{figure}

\begin{figure}[h!]
\centering
\includegraphics[scale=0.5]{Caso3t2.jpg}
\caption{Caso 3 t=3/4 T}
\label{fig:11}
\end{figure}

\begin{figure}[h!]
\centering
\includegraphics[scale=0.5]{Caso3t.jpg}
\caption{Caso 3 t= T}
\label{fig:12}
\end{figure}

Las anteriores gráficas presenta el comportamiento para condiciones de frontera periódicas, Tarda mas interacciones que las otras condiciones. Su comportamiento indica que plato no llega a un equilibrio continuo simétrico circular, sino que partes intermedias entre los bordes se calientan mas que los bordes. .\\

\begin{figure}[h!]
\centering
\includegraphics[scale=0.5]{Caso3Tprom.jpg}
\caption{Caso 3 temperatura promedio VS tiempo}
\label{fig:15}
\end{figure}

Se generaron animaciones movie.gif para caso1, movie2.gif para caso 2, movie3.gif para el caso 3, están en la carpeta de los archivos.

\end{document}

